\documentclass[a4paper,twoside]{article}

\usepackage{epsfig}
\usepackage{subfigure}
\usepackage{calc}
\usepackage{amssymb}
\usepackage{amstext}
\usepackage{amsmath}
\usepackage{amsthm}
\usepackage{multicol}
\usepackage{pslatex}
\usepackage{apalike}
\usepackage{SCITEPRESS}     % Please add other packages that you may need BEFORE the SCITEPRESS.sty package.

\subfigtopskip=0pt
\subfigcapskip=0pt
\subfigbottomskip=0pt

\begin{document}

\title{EvoloPy: An open source Nature-Inspired Optimization Toolbox in Python}

\author{\authorname{First Author Name\sup{1}, Second Author Name\sup{1} and Third Author Name\sup{2}}
\affiliation{\sup{1}Institute of Problem Solving, XYZ University, My Street, MyTown, MyCountry}
\affiliation{\sup{2}Department of Computing, Main University, MySecondTown, MyCountry}
\email{\{f\_author, s\_author\}@ips.xyz.edu, t\_author@dc.mu.edu}
}

\keywords{The paper must have at least one keyword. The text must be set to 9-point font size and without the use of bold or italic font style. For more than one keyword, please use a comma as a separator. Keywords must be titlecased.}

\abstract{EvoloPy is an open source and cross-platform Python toolbox that implements a wide range of classical and recent nature-inspired metaheuristic algorithms. The goal of this toolbox is to facilitate the use of metaheuristic algorithms by non-specialists coming from different domains. With a simple interface and minimal dependencies, it is easier for researchers and practitioners to utilize EvoloPy for optimizing and benchmarking their own defined problems using the most powerful metaheuristic optimizers in the literature. On the other hand, the design of this toolbox makes it very easy for the researchers in the domain to integrate their own optimizers and compare their performance to the state of art algorithms. }


\onecolumn \maketitle \normalsize \vfil
\section{\uppercase{Introduction}}

\label{sec:introduction}
testttt







\section*{\uppercase{Acknowledgements}}




\vfill
\bibliographystyle{apalike}
{\small
\bibliography{ref}}


\section*{\uppercase{Appendix}}

\noindent If any, 
\textit{$\backslash$section*\{APPENDIX\}}

\vfill
\end{document}

